This work consists in solving the two main problems of ``k-means''-like
algorithms described in Section \ref{related} and \ref{problem_definition}:
choosing the $k$ parameter and placing the $k$ initial centroids.
The only assumption is that the dataset is generated from one or more gaussian
disitributions.


\subsection{The choice of $k$}
The most common scenario does not involve a domain expert. As a matter of fact,
no prior knowledge can be used to guess a proper and reasonable value for $k$.\\
The main idea behind the solution proposed in this work is that of exploiting global
and local density analysis to get an initial estimate of the number of clusters
in the data.

More formally, a \emph{peak detection} procedure is used to retrieve
all the peaks (i.e. local maxima) in the density function for each feature\footnote{If
the item considered is a point in an n-dimensional euclidean space, this is equivalent to
finding all the local maxima in the density functions for each axis.} of the items in
the dataset. In this way, a \emph{peak} is a data sample that is larger than
its two neighbour samples.
This results in a list of peaks for each feature space and finally a n-dimensional grid
matching every possible peak among all the features available is created
(see Algorithm \ref{alg:grid_creation}).
\begin{algorithm}[h]
 \KwIn{xs array: points'x values}
 \KwIn{ys array: points'y values}
 \KwResult{array: initial centroids}

 hx = computeMarginalDensity(xs)\;
 hy = computeMarginalDensity(ys)\;

 px = detectPeaks(hx)\;
 py = detectPeaks(hy)\;

 centroids := []\;
 \For{i in px}{
    \For{j in py}{
        centroids += (hx.x[i], hy.x[j])\;
    }
 }
% return [(hx[x], hy[y]) for x in px for y in py]
 \KwRet{centroids}
 \label{alg:grid_creation}
 \caption{Initial centroids identification}
\end{algorithm}


As Figure~\ref{grid} shows, the black diamonds on the top and on
the left sub-figures are the peaks found on the density distribution function among all
the points in the dataset, while the orange points in the central sub-figure represent the
grid of all the potential centroids.
In this way, the worst-case scenario is taken into account, allowing this approach not
to miss any potential centroid in the whole dataset\footnote{This approach produces an
over-estimation of the real number of centroids, introducing possible false-negatives.}.

\begin{figure*}
  \center{\includegraphics[width=0.7\textwidth]{unbalance_rescaled_coolfig.png}}
  \caption{The grid with all the potential centroids computed from the peaks.}
  \label{grid}
\end{figure*}

Moreover, among all the potential peaks created, only those which have at least one data
item assigned after a run of the k-means are kept. This allows to discard all the
false-positives that were found creating all the possible combinations in all the feature spaces.
Furthermore, it represents the refinement step for the algorithm.

Finally, the items in this grid are the initial centroids, representing both the value for the
$k$ parameter (i.e. the number of items in this grid) and their coordinates.


\subsection{Centroids refinement}
The other important phase is to find a good positioning of the $k$ initial centroids.
The main idea behind this feature is the following:
\begin{enumerate}
    \item \label{step1} use peaks' locations as the ones of the centroids.
    \item build an ellipse around every centroid with the $(a,b)$ parameters (namely the x-axis
        and the y-axis radii) applying the formula in Equation \ref{ellipse_params} (see
        Algorithm~\ref{alg:find_ellipses}).
    \item merge all the ellipses that intersect with each other (see
        Algorithm~\ref{alg:merge_procedure}).
    \item go to \ref{step1} until convergence or some other exit criterion is met.
\end{enumerate}

\begin{algorithm}[h]
    \KwIn{centroids, [(centroidId,x,y)]}
    \KwIn{clusters, [\{id: [(x, y)]\}]}
    \KwResult{ellipses}

    cDesnity = {}\;
    $dx$ = $dy$ = []\;
    \For{cluster $\in$ clusters}{
        $(x_{\mu}, x_{\sigma})$ = normFit (cluster.x)\;
        $(y_{\mu}, y_{\sigma})$ = normFit (cluster.y)\;

        $dx$ += px = normPDF $(x_{\mu}, x_{\mu}, x_{\sigma})$\;
        $dy$ += py = normPDF $(y_{\mu}, y_{\mu}, y_{\sigma})$\;

        cDensity[cluster.id] = $((x_{\mu}, x_{\sigma}, px), (y_{\mu}, y_{\sigma}, py))$\;
    }

    ellipses = []\;
    \For{$(c, ((x_x, x_y, x_p), (y_x, y_y, y_p))) \in$ cDensity}{
        ellipses += $(c$,\\
            $((x_x, x_y, (x_p - dx_{\mu})/dx_{\sigma})$, \\
            $(y_x, y_y,  (y_p - dy_{\mu})/dy_{\sigma})))$\;
    }

    \KwRet{ellipses}
    \caption{Compute Estimated Influence Area}\label{alg:find_ellipses}
\end{algorithm}

\begin{algorithm}[h]
    \KwIn{ellipses}
    \KwIn{cstats, \{id: (xsum, ysum, n)\}}
    \KwResult{centroids}

~\\
    // Fing merges
    merges = []
    \For{$e_0, e_1 \in ellipses$}{
        // Compute $a,b$ ellipse paramteres (Eq~\ref{ellipse_params})\\
        $eia_0$ = computeEIA ($e_0$)\;
        $eia_1$ = computeEIA ($e_1$)\;

        \If{$Ellipse(e_0, eia_0) \cap Ellipse(e_1, eia_1) \neq \varnothing$}{
            merges += $(e_0, e_1)$\;
        }
    }

~\\
    // Apply merges
    merged = {}\;
    \For{($e_0, e_1) \in merges$}{
        $e_0$ = findCurrentIdIfMerged ($e_0$, merged)\;
        $e_1$ = findCurrentIdIfMerged ($e_1$, merged)\;

        $(xsum_0, ysum_0, n_0)$ = cstats[$e_0$.id]\;
        $(xsum_1, ysum_1, n_1)$ = cstats[$e_1$.id]\;

        cstats[$e_0$.id] = \\
            $(xsum_0 + xsum_1, ysum_0 + ysum_1, n_0 + n_1)$\;
        merged[$e_1$.id] = $e_0$.id\;
    }

~\\
    // Derive centroids
    centroids = []\;
    \For{(c, (xsum, ysum, n)) $\in$ cstats}{
        centroids += $(xsum/n, ysum/n)$\;
    }

    \KwRet{centroids}
    \caption{Merge centroids with overlapping EIAs}\label{alg:merge_procedure}
\end{algorithm}


The formula described in Equation~\ref{ellipse_params} computing the heigth and the width of
the ellipses is the key aspect of the proposed merging strategy.
\begin{equation}
\label{ellipse_params}
    f(\sigma, cdens) = ((\sigma * 2 * 0.35) + (cdens * 0.7)) * 5
\end{equation}
Where:
\begin{itemize}
    \item $\sigma$ is the standard deviation of the gaussian distribution underlying the cluster.
    \item $cdens$ is the value of the gaussian distribution Probability Distribution Function (PDF)
        in the mean.
\end{itemize}

To better understand the reasoning behind this formuls, a more precise explenation of its components
is needed.\\
Given the mean and the standard deviation of a cluster,
$\sigma$ is doubled to take both the sides of the gaussian distribution into account.
The 0.35 and the 0.7 values allow the density in the mean value of the gaussian distribution ($cdens$)
to influence more the size of the ellipse rather than its standard deviation ($\sigma$). Finally, 5 is
an overall scaling factor useful to allow the comparison of cluster with very different densities.

From an higher point of view, Equation \ref{ellipse_params} represents the \emph{Estimated Influence Area}
(EIA) for each cluster. As the name suggests, this depicts (also visually) the area of influence a
cluster has on all the others.\\
As it is possible to see from Figure~\ref{start}, the ellipses represent all the initial
\emph{EIAs} starting from all the potential centroids after discarding
all the centroids that has no item belonging to them (which are clearly false-positives).
If two clusters' estimated influence area has a non-empty intersection, it
means that they can be collapsed and become a single bigger cluster.
The Figures~\ref{start}, \ref{middle} and \ref{end} depicts the trend and the position of the
\emph{EIAs} during an example merging procedure.

Moreover, as it is possible to notice from Figure \ref{end}, the approach described in this
section started with many potential centroids (see Figure~\ref{start}) and after a
few iterations converged finding the 8 clusters in the dataset successfully
(see Figure~\ref{end}).

\begin{figure}[t]
  \center{\includegraphics[width=0.5\textwidth]{unbalance_rescaled_density_0.png}}
  \caption{The first estimated influence areas after filtering some of the false-positives.}
  \label{start}
\end{figure}

\begin{figure}[t]
  \center{\includegraphics[width=0.5\textwidth]{unbalance_rescaled_density_3.png}}
  \caption{The situation after 4 iterations.}
  \label{middle}
\end{figure}

\begin{figure}[t]
  \center{\includegraphics[width=0.5\textwidth]{unbalance_rescaled_density_6.png}}
  \caption{The final outcome of the algorithm.}
  \label{end}
\end{figure}

A noteworthy aspect is that all the aforementioned factors used inside Equation \ref{ellipse_params} have
been set after several tuning stages through empirical tests using many datasets with different
clusters' properties.

Nevertheless, representing the clusters by means of sufficient statistics allows to implement an $O(1)$
merging procedure. More in detail, every cluster is internally represented as a touple with the
following information:
\begin{equation*}
    \left(\left[\sum_{p}^{|C|} feat[0],\dots,\sum_{p}^{|C|} feat[n]\right],|C|\right)
\end{equation*}
In a 2-dimensional euclidean space, this metadata result in a list containing the the sum of all the
points' coordinates among the 2 axes and the number of points within the cluster taken into consideration.
This structure contains the least possible information to compute the centroid's coordinates as the
barycenter of that particolar cluster.

Finally, this merging strategy enables all the \emph{Data Mining Desiderata} mentioned in Section
\ref{intro}. It makes use of sufficient statistics to internally represent every cluster, hence
requiring an overall amount of memory that is linear in the number of peaks ($O(k)$, \emph{limited memory}
property). Furthermore, the \emph{online} behaviour is guaranteed by default, since the centroid
bootstrap and merging procedure always return a solution at any point in time. Finally, it allows
to deal with \emph{straming} data thanks both to its linear memory consumption and to its ability to
work with chunks of data by default, propagating all the sufficient statistics needed to update the
centroids from one iteration to another. This also guarantees that if the computation is stopped
it can be restarted without the need to re-process everything from scratch.


\subsection{Applications}
The approach presented in this work can have two main applications. On the one hand,
this procedure can be used as a bootstrapping phase for a  ``k-means''-like algorithm.
On the other, it can be integrated with other partitional clustering algorithms to
refine the local solution during the computation phase, allowing the system as a
whole to find increasingly better clusters.

Moreover, this work is agnostic with respect
to the metric used to compute the distance between the items. For the seek of
completeness, all the figures presented in this work are computed using a
standard euclidean distance.

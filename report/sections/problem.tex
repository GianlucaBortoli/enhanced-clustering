Clustering is the task of gathering items in a way that elements belonging
to the same group (the \emph{cluster}) are more similar to each other other than the ones
assigned to the others.\\
More formally, given a input:
\begin{itemize}
    \item $X = \{x_0, \dots ,x_n\}$, the initial set of elements.
    \item $d: X \times X \to \mathbb{R}$, a \emph{metric} measuring the similarity.
\end{itemize}
The final goal is to find the cluster configuration
\begin{equation*}
    C = \left\{ c_0, \dots , c_m \right\} \mid \bigcup_{C} = X
\end{equation*}
partitioning $X$ into $m$ clusters, maximizing the intra-cluster distance
(dual problem of minimizing inter-cluter distance):

\begin{equation}
    \underset{C}{\mathrm{argmax}}
    \sum_{c \in C}
        \sum_{i,j}^{|c|}
            d(c_i,c_j)
\end{equation}

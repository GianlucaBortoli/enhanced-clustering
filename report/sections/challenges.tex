The concept of clustering is simple and powerful; moreover, its versatility
and generality are its greatness and at the same time source of many difficulties.

%% Metrics identification

Given the only strict requirement to be the definition of a metric over the
domain of $X$, clustering is applied to a wide variety of problems.
Clearly, each domain offers different optimization opportunities and
particular challenges.
In particular, the choice of the metric heavily influences the final outcome quality.
As a result even the medical~\cite{siless2013comparison}, the mechanical
engineering~\cite{wilding2011clustering} and the mobile networks~\cite{cheng2009stability}
literatures features different studies that address this particular challenge
suggesting highly specialized distance functions.


%% Inter/Intra cluster distance measure
Once the proper metric is identified, the following huge influencing factors
are the mathematical definitions of ``intra-cluster'' and ``inter-cluster''
distances. They vary a lot in the different implementation leading to
completely different clustering configurations.
For example, when performing agglomerative clustering three methods
are widely used to define the distance between two clusters:
average, minimum, and maximum distance.
The average approach uses the barycenters, while the minimum (maximum)
relies upon the minimal (maximal) distance between any two points
belonging to a different clusters.



